\begin{center}
\emph{\LARGE\textbf {Abstract}}\\[2.5cm]
\end{center}


\doublespacing
Agriculture plays a pivotal role in human’s life as it the only source of livestock, along with that it also provides employment opportunities and plays a major role in country’s economy, hence it is important to maintain standards in production quality. Recent advancements in technology have led to the hybridization of agriculture and machine learning methodologies thus helping in the improvement of crop quality. However it is observed from the literature survey that most of the work focuses on detection of crop damage and not on cause of damage, hence in this work an attempt is made to identify the cause of damage especially considering the case of pesticide usage. For this experiment the dataset of around 1.48 lakh samples has been used to train and predict the cause of crop damage with the help of some of the well know machine learning models. The results show that LGBM performs better than other models both in terms of classification accuracy and speed. 
\vspace*{4cm}



